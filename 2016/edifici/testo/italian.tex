\usepackage{xcolor}
\usepackage{afterpage}
\usepackage{pifont,mdframed}
\usepackage[bottom,symbol]{footmisc}

\makeatletter
\gdef\this@inputfilename{input.txt}
\gdef\this@outputfilename{output.txt}
\makeatother

\createsection{\Input}{Dati di input}
\createsection{\Output}{Dati di output}

\newcommand{\inputfile}{\texttt{input.txt}}
\newcommand{\outputfile}{\texttt{output.txt}}

% % % % % % % % % % % % % % % % % % % % % % % % % % % % % % % % % % % % % % % % % % %
% % % % % % % % % % % % % % % % % % % % % % % % % % % % % % % % % % % % % % % % % % %

Luca ha iniziato a studiare in università. La struttura del grande campus universitario è particolarmente complessa. Esso è infatti costituito da $N$ edifici dove vengono svolte diverse attività. In particolare $K$ di questi $N$ edifici contengono aule di didattica e aule studio, mentre i rimanenti $N-K$ contengono alloggi per gli studenti.

Gli edifici sono tra loro connessi grazie a $M$ strade. Ciascuna ha una lunghezza, espressa in metri, e collega due edifici. Tutte le strade possono essere percorse in entrambi i sensi.

Dopo qualche mese in cui Luca si è spostato dalla sua città natale all'università in treno, ha deciso che è giunto il momento di affittare un alloggio in uno degli edifici dell'università. In tal modo sarà più vicino all'università e potrà alzarsi più tardi la mattina.

Lungo una giornata tipica, Luca va a trovare i suoi amici che sono dislocati a gruppetti in tutti i $K$ edifici che contengono aule studio. La mattina Luca esce, fa il giro degli edifici in cui ci sono i suoi amici e al termine fa rientro nella sua stanza.

In quale edificio (adibito ad alloggi) conviene a Luca affittare una stanza per minimizzare la distanza percorsa ogni giorno, sapendo che può andare a visitare gli amici nell'ordine che vuole?

\Input
Il file \inputfile{} è composto da $2+M$ righe. La prima riga contiene tre interi $N$, $M$, $K$ separati da spazio. Essi sono rispettivamente: il numero di edifici, il numero di strade e il numero di edifici contenenti aule studio.
La seconda riga contiene $K$ interi, i numeri identificativi degli edifici contenenti aule studio.
Seguono $M$ righe che descrivono le strade: ognuna contiene tre numeri $A$, $B$, $L$ ad indicare che esiste una strada tra gli edifici $A$ e $B$ di lunghezza $L$.

\Output
Il file \outputfile{} contiene un singolo intero: la distanza minima che Luca deve percorrere in un giorno, facendo la miglior scelta per l'affitto della stanza.

\Constraints
\begin{itemize}[nolistsep, itemsep=2mm]
\item $2 \le N \le 5\,000$.
\item $1 \le M \le 100\,000$.
\item $1 \le K \le 6$.
\item $K < N$.
\item Una stessa strada viene indicata al massimo una volta nel file di input.
\item Presi due edifici qualunque, esiste sempre almeno un percorso che li collega.
\item $1 \le A_i, B_i \le N$ per ogni $i=0 \ldots N-1$.
\item $1 \le L_i \le 1\,000$ per ogni $i=0 \ldots N-1$.
\item Il percorso giornaliero può passare anche più di una volta lungo una stessa strada.
\end{itemize}

\Scoring
Il tuo programma verrà testato su diversi test case raggruppati in subtask.
Per ottenere il punteggio relativo ad un subtask, è necessario risolvere
correttamente tutti i test relativi ad esso.
\begin{itemize}[nolistsep,itemsep=2mm]
\item \textbf{\makebox[2cm][l]{Subtask 1} [5 punti]}: Casi d'esempio.
\item \textbf{\makebox[2cm][l]{Subtask 2} [30 punti]}: $K = 1$, ovvero Luca parte dalla sua stanza per raggiungere l'unico gruppo di amici e torna subito indietro.
\item \textbf{\makebox[2cm][l]{Subtask 3} [15 punti]}: $K = 2$.
\item \textbf{\makebox[2cm][l]{Subtask 4} [30 punti]}: $N \le 500$.
\item \textbf{\makebox[2cm][l]{Subtask 5} [20 punti]}: Nessuna limitazione specifica.
\end{itemize}

\Examples
\begin{example}
\exmpfile{edifici.input0.txt}{edifici.output0.txt}%
\exmpfile{edifici.input1.txt}{edifici.output1.txt}%
\end{example}

\Explanation

\begin{figure}[h!]
    \centering
    \includegraphics[width=2in]{edifici_asy/example.pdf}
\end{figure}


Nel \textbf{secondo caso di esempio}, rappresentato in figura, una scelta ottimale è affittare una stanza nell'edificio 1. A quel punto il percorso giornaliero di Luca consiste nel partire dall'edificio 1, percorrere una strada di lunghezza 2 per arrivare all'edificio 3, quindi una strada di lunghezza 3 per arrivare all'edificio 5, infine una strada di lunghezza 4 per tornare al proprio alloggio.
