\usepackage{xcolor}
\usepackage{afterpage}
\usepackage{pifont,mdframed}
\usepackage[bottom,symbol]{footmisc}

\makeatletter
\gdef\this@inputfilename{input.txt}
\gdef\this@outputfilename{output.txt}
\makeatother

\createsection{\Input}{Dati di input}
\createsection{\Output}{Dati di output}

\newcommand{\inputfile}{\texttt{input.txt}}
\newcommand{\outputfile}{\texttt{output.txt}}

% % % % % % % % % % % % % % % % % % % % % % % % % % % % % % % % % % % % % % % % % % %
% % % % % % % % % % % % % % % % % % % % % % % % % % % % % % % % % % % % % % % % % % %

Il professor Tarboto sta pensando come gestire le verifiche nel prossimo anno scolastico. Dopo un'accurata riflessione, ha deciso che opererà nel modo seguente: organizzerà $N$ verifiche lungo l'anno e in ciascuna di queste chiederà esattamente $K$ argomenti.

I suoi furbi studenti sono venuti in possesso della pianificazione e conoscono esattamente in quali giorni il professore ha pianificato le verifiche. Tali giorni sono numerati a partire da 0, che equivale al primo giorno di scuola.

Sfortunatamente, la memoria degli studenti è limitata e il loro carico di lavoro deve essere umano: questo significa che in un giorno uno studente può studiare un solo argomento, che conoscerà quindi a partire dal giorno seguente. Inoltre si ricorderà tale argomento, indipendentemente da quale sia, per $T$ giorni. Ad esempio: se uno studente studia un argomento il giorno 2, lo conoscerà dal giorno 3 al giorno $2 + T$ (incluso).

Gli studenti possono studiare anche prima dell'inizio dell'anno scolastico, se necessario, con le stesse modalità esposte sopra.

Per ottimizzare la pianificazione aiuta gli studenti a capire qual è il numero minimo di giorni in cui devono studiare, affinché in occasione di \emph{tutte} le verifiche lungo l'anno sappiano \emph{almeno} $K$ argomenti!


\Input
Il file \inputfile{} è composto da $2$ righe. La prima riga contiene tre interi $N$, $T$, $K$ separati da spazio. Essi sono rispettivamente: il numero di verifiche, il numero di giorni per cui uno studente memorizza un argomento e il numero di argomenti richiesti ad ogni verifica.
La seconda riga contiene $N$ interi, i giorni (a partire da 0) in cui saranno svolte le verifiche.

\Output
Il file \outputfile{} contiene un singolo intero: il numero minimo di giorni in cui uno studente deve studiare per arrivare preparato alle verifiche. Se non è possibile rispettare le condizioni del problema, stampare -1.

\pagebreak

\Constraints
\begin{itemize}[nolistsep, itemsep=2mm]
\item $1 \le N, T, K \le 1\,000$.
\item Un anno scolastico ha al più $2\,000$ giorni, quindi i giorni di svolgimento delle verifiche sono compresi tra 0 e $1\,999$.
\item I giorni di svolgimento delle verifiche sono forniti in ordine cronologico.
\end{itemize}
\Scoring
Il tuo programma verrà testato su diversi test case raggruppati in subtask.
Per ottenere il punteggio relativo ad un subtask, è necessario risolvere
correttamente tutti i test relativi ad esso.
\begin{itemize}[nolistsep,itemsep=2mm]
\item \textbf{\makebox[2cm][l]{Subtask 1} [10 punti]}: Casi d'esempio.
\item \textbf{\makebox[2cm][l]{Subtask 2} [40 punti]}: $N = 1$, ovvero c'è solo una verifica in tutto l'anno.
\item \textbf{\makebox[2cm][l]{Subtask 3} [50 punti]}: Nessuna limitazione specifica.
\end{itemize}

\Examples
\begin{example}
\exmpfile{verifiche.input0.txt}{verifiche.output0.txt}%
\exmpfile{verifiche.input1.txt}{verifiche.output1.txt}%
\end{example}
