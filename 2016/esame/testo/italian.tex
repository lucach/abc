\usepackage{xcolor}
\usepackage{afterpage}
\usepackage{pifont,mdframed}
\usepackage[bottom,symbol]{footmisc}

\makeatletter
\gdef\this@inputfilename{input.txt}
\gdef\this@outputfilename{output.txt}
\makeatother

\createsection{\Input}{Dati di input}
\createsection{\Output}{Dati di output}

\newcommand{\inputfile}{\texttt{input.txt}}
\newcommand{\outputfile}{\texttt{output.txt}}

% % % % % % % % % % % % % % % % % % % % % % % % % % % % % % % % % % % % % % % % % % %
% % % % % % % % % % % % % % % % % % % % % % % % % % % % % % % % % % % % % % % % % % %

Anche quest'anno è giunta l'ora degli esami di maturità e il presidente della commissione della classe di Tommaso ha già stabilito regole severe per lo svolgimento della prova orale, alla presenza dei vari docenti.

Il presidente ha deciso che la disposizione dei tavoli, dietro a cui siederanno i docenti, sarà a forma di cerchio (con il candidato al centro). Ha comunicato inoltre che intende adottare la seguente prassi:

\begin{itemize}
\item Ciascun docente ha diritto a porre un'unica domanda, caratterizzata da una certa difficoltà $D$.
\item Un docente non può porre la sua domanda se uno dei suoi due vicini è già intervenuto per la sua domanda. Chiaramente, nella disposizione a cerchio consideriamo vicini le due persone sedute (alla propria destra e alla propria sinistra).
\end{itemize}

Sapendo che la difficoltà complessiva di un esame è data dalla somma delle difficoltà delle domande poste, qual è la \emph{massima} difficoltà ottenibile rispettando quanto chiesto dal presidente?

\Input
Il file \inputfile{} è composto da $2$ righe. La prima riga contiene un intero $N$, il numero dei docenti. La seconda riga contiene $N$ interi, ciascuno rappresentante la difficoltà della domanda dell'$i$-esimo docente.

\Output
Il file \outputfile{} contiene un singolo intero: la massima difficoltà ottenibile per un esame, rispettando le richieste del presidente.

\Constraints
\begin{itemize}[nolistsep, itemsep=2mm]
\item $2 \le N \le 1\,000$.
\item $1 \le D_i \le 100$ per ogni $i=0 \ldots N-1$.
\item Essendo la disposizione a cerchio, il docente numero $0$ si considera seduto a fianco del docente $N-1$ (e viceversa).
\end{itemize}

\pagebreak

\Scoring
Il tuo programma verrà testato su diversi test case raggruppati in subtask.
Per ottenere il punteggio relativo ad un subtask, è necessario risolvere
correttamente tutti i test relativi ad esso.
\begin{itemize}[nolistsep,itemsep=2mm]
\item \textbf{\makebox[2cm][l]{Subtask 1} [10 punti]}: Casi d'esempio.
\item \textbf{\makebox[2cm][l]{Subtask 2} [50 punti]}: $N \le 20$.
\item \textbf{\makebox[2cm][l]{Subtask 3} [40 punti]}: Nessuna limitazione specifica.
\end{itemize}

\Examples
\begin{example}
\exmpfile{esame.input0.txt}{esame.output0.txt}%
\exmpfile{esame.input1.txt}{esame.output1.txt}%
\end{example}

\Explanation

Nel \textbf{primo caso di esempio} la massima difficoltà è ottenuta dalle domande dei docenti 0, 2 e 4 (per un totale di $8 + 4 + 6 = 18$). Si osservi che $8 + 5 + 9 = 22$ porta a un totale più alto ma non è una risposta lecita, perché le domande di difficoltà 8 e 9 arrivano da due docenti seduti vicini.

Nel \textbf{secondo caso di esempio} ci sono diverse combinazioni di domande che danno luogo alla massima difficoltà ma non c'è un modo di sceglierle, rispettando le regole, la cui somma superi 2.
