
\documentclass[a4paper,11pt]{article}

\usepackage[utf8x]{inputenc}
\SetUnicodeOption{mathletters}
\SetUnicodeOption{autogenerated}

\usepackage{lmodern}
\renewcommand*\familydefault{\sfdefault} %% Only if the base font of the document is to be sans serif
\usepackage[T1]{fontenc}

\usepackage[italian]{babel}
\usepackage{booktabs}
\usepackage{mathpazo}
\usepackage{graphicx}
\usepackage[left=2cm, right=2cm, bottom=3cm]{geometry}
\frenchspacing

\begin{document}
\noindent {\Large ABC 2014}
\vspace{0.5cm}

\noindent{\Huge Previsioni meteorologiche (\texttt{previsioni})}


\section*{Descrizione del problema}

Il colonnello Bernacca ha studiato a lungo il tempo meteorologico del paese di Tailia e ha scoperto le regole per prevedere se in un certo giorno piove (P) o c’è il sole (S). La previsione funziona a partire dal giorno della creazione e funziona all’infinito per un numero di giorni qualsiasi. La sequenza del tempo su Tailia è iniziata così:
\begin{verbatim}
P S S P S S S P S S P S S S S P S S P S S S P S S P S S S S S 
\end{verbatim}
\noindent
e significa che il giorno della creazione su Tailia ha piovuto, il secondo e il terzo c’è stato il sole e così via.\\
Per ricavare il tempo in un certo giorno si possono usare una serie di regole che permettono di costruire previsioni lunghe a piacere, secondo la seguente regola: la \textbf{K-esima} previsione, si ottiene prendendo la previsione \textbf{K-1}, seguita da un giorno di pioggia (P) e da \textbf{K+2} giorni di sole (S), seguiti di nuovo dalla previsione \textbf{K-1}.\\
Le prime previsioni risultano quindi:
\begin{itemize}
\item Previsione 0 = \texttt{P S S} 
\item Previsione 1 = \texttt{P S S P S S S P S S}
\item Previsione 2 = \texttt{P S S P S S S P S S P S S S S P S S P S S S P S S}
\end{itemize}
e così via...\\
Aiuta il colonnello Bernacca, che dato un giorno a partire dal giorno della creazione di Tailia, vuole sapere se, seguendo il suo modello di previsioni, in quel giorno pioverà (P) o ci sarà il sole (S).


\section*{Dati di input}
L'input è costituito da un unico numero intero positivo \textbf{N}.


\section*{Dati di output}
  
L'output è costituito da un unico carattere, P o S: se il giorno N pioverà (P) o ci sarà il sole (S).
    
  \section*{Assunzioni}
$1 \leq N \leq 10^9.$

\section*{Esempi di input/output}

  
    \noindent
    \begin{tabular}{p{8cm}|p{8cm}}
    \toprule
    \textbf{Input \texttt{(stdin)}}
    & \textbf{Output \texttt{(stdout)}}
    \\
    \midrule
    \small
    \begin{verbatim}
2
      \end{verbatim}
    &
    \small
    \begin{verbatim}
S
\end{verbatim}
    \\
    \bottomrule
    \end{tabular}
    \\[12pt]
        \noindent
    \begin{tabular}{p{8cm}|p{8cm}}
    \toprule
    \textbf{Input \texttt{(stdin)}}
    & \textbf{Output \texttt{(stdout)}}
    \\
    \midrule
    \small
    \begin{verbatim}
11
      \end{verbatim}
    &
    \small
    \begin{verbatim}
P
\end{verbatim}
    \\
    \bottomrule
    \end{tabular}
  

\end{document}
