
\documentclass[a4paper,11pt]{article}

\usepackage[utf8x]{inputenc}
\SetUnicodeOption{mathletters}
\SetUnicodeOption{autogenerated}

\usepackage[italian]{babel}

\usepackage{lmodern}
\renewcommand*\familydefault{\sfdefault} %% Only if the base font of the document is to be sans serif
\usepackage[T1]{fontenc}

\usepackage{booktabs}
\usepackage{mathpazo}
\usepackage{graphicx}
\usepackage[left=2cm, right=2cm, bottom=3cm]{geometry}
\frenchspacing

\begin{document}
\noindent {\Large ABC 2014}
\vspace{0.5cm}

\noindent{\Huge Messaggi (\texttt{messaggi})}


\section*{Descrizione del problema}
    Luca sta preparando un sistema di messaggistica semplice per consentire la comunicazione tra gli studenti (soprattutto durante le verifiche, un sistema simile torna sempre utile). \\
    Luca ha finito di realizzare il sistema e lo sta provando: purtroppo qualcosa non funziona e bisogna cercare di capire la situazione. \\
    Da buon sistemista, la prima cosa che fa è analizzare il file di log conservato sul server che registra ogni singolo messaggi scambiato tra gli utenti. \\
    Un utente è identificato con nome (con lunghezza massima 10 caratteri tra lettere minuscole e numeri). \\
    Dopo un po' di analisi, Luca ha capito che il problema non risiede nel server: \textbf{tutti i messaggi sono memorizzati correttamente}. Probabilmente quindi è l'app che ha realizzato per gli smartphone dei suoi compagni ad avere qualche problema. Per verificare ciò, aiutalo a rispondere a due tipi di domande:
    \begin{itemize}
        \item{\verb|utente INVIATI| \\che deve restituire l'elenco degli utenti a cui $utente$ ha inviato un messaggio}
        \item{\verb|utente RICEVUTI|\\ che deve restituire l'elenco degli utenti da cui $utente$ ha ricevuto un messaggio}
    \end{itemize}
    Così come il file di log è ordinato cronologicamente, allo stesso modo gli elenchi dell'utente dovranno essere nello stesso ordine con cui gli utenti compaiono nel log. 


\section*{Dati di input}
  
L'input è composto da 1 + N + R righe. \\
La prima riga contiene due interi, N e R, rispettivamente il numero di righe del file di log (ovvero, quanti messaggi sono stati scambiati) e il numero di richieste a cui si deve rispondere.\\ \\
Le successive N righe sono nella forma
\begin{verbatim}
mittente destinatario
\end{verbatim} e indicano che un messaggio è stato inviato da $mittente$ a $destinatario$.\\ \\
Le ultime R righe sono nella forma
\begin{verbatim}
utente TIPORICHIESTA
\end{verbatim}
indicando che Luca vuole sapere l'elenco degli utenti da cui $utente$ ha ricevuto un messaggio (nel caso di \verb|RICEVUTI|) oppure l'elenco degli utenti a cui $utente$ ha inviato un messaggio (nel caso di \verb|INVIATI|)

\section*{Dati di output}
  
L'output deve essere formato da $R$ righe.\\
Ciascuna di queste righe deve contenere un intero non negativo $U_i$, il numero di utenti che corrispondono alla R-esima richiesta, seguito dagli $i$ nomi degli utenti.
    
  \section*{Assunzioni}

$N, R \leq 100000$. \\Nel 50\% dei casi vale che $N, R \leq 1000$.
    
\section*{Esempi di input/output}

  
    \noindent
    \begin{tabular}{p{8cm}|p{8cm}}
    \toprule
    \textbf{Input \texttt{(stdin)}}
    & \textbf{Output \texttt{(stdout)}}
    \\
    \midrule
    \small
    \begin{verbatim}
3 2
luca mario
luca pietro
francesca pietro
luca INVIATI
pietro RICEVUTI

      \end{verbatim}
    &
    \small
    \begin{verbatim}
2 mario pietro
2 luca francesca
      \end{verbatim}
    \\
    \bottomrule
    \end{tabular}
  
\section*{Note}
\begin{itemize}
  
    \item{Più messaggi possono essere inviati dallo stesso mittente anche verso lo stesso destinatario. Tutti i messaggi dovranno comparire nell'output del problema.}
    \item{In ogni messaggio, il mittente \textbf{non coincide mai} con il destinatario.}
\end{itemize}



\end{document}

