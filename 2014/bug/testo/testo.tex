
\documentclass[a4paper,11pt]{article}

\usepackage[utf8x]{inputenc}
\SetUnicodeOption{mathletters}
\SetUnicodeOption{autogenerated}

\usepackage[italian]{babel}

\usepackage{lmodern}
\renewcommand*\familydefault{\sfdefault} %% Only if the base font of the document is to be sans serif
\usepackage[T1]{fontenc}

\usepackage{booktabs}
\usepackage{mathpazo}
\usepackage{graphicx}
\usepackage[left=2cm, right=2cm, bottom=3cm]{geometry}
\frenchspacing

\begin{document}
\noindent {\Large ABC 2014}
\vspace{0.5cm}

\noindent{\Huge Bug (\texttt{bug})}


\section*{Descrizione del problema}
Il professor Tarboto sta per terminare di scrivere il suo nuovo libro di reti e sistemi informatici, ma si è accorto di essere un po' indietro con il lavoro: la fine dell'anno scolastico si sta avvicinando e la casa editrice vuole che il lavoro sia assolutamente pronto entro giugno. \\
In particolare, restano da sistemare ancora \textbf{M} bug nei codici degli algoritmi presenti nel libro. \\
Grazie alla sua proverbiale esperienza, il professor Tarboto decide quindi di chiedere la collaborazione dei suoi \textbf{N} studenti affinché risolvano i problemi al suo posto. Ovviamente, gli studenti \textbf{possono lavorare in parallelo}. \\
Gli studenti del professore però, saputo dell'urgenza con cui il loro docente deve finire il lavoro, chiedono in cambio per l'essere impiegati (\textbf{indipendentemente dal volume di lavoro}, ovvero da quanti bug devono risolvere) un certo numero di punti (P) in più nella prossima verifica di Sistemi e Reti. \\
Il professor Tarboto è disposto a fare questa concessione purché il \textbf{totale} dei punti aggiuntivi che dovrà regalare ai suoi studenti non ecceda la somma \textbf{S} (non vuole che tutti prendano 10 alla prossima verifica... il preside potrebbe insospettirsi!). \\
Il professore sa che i bug, come gli studenti, non sono tutti uguali: ogni bug è caratterizzato dall'avere una complessità (C) e può essere risolto \textbf{in un giorno} (indipendentemente dalla complessità) da uno studente con un'abilità (A) tale che \textbf{$A \geq C$}.\\
Aiuta il professor Tarboto a stabilire a quale studente assegnare ciascun bug in modo da risolverli tutti \textbf{nel minor numero di giorni} possibile.

\section*{Dati di input}
  
L'input è composto da 4 righe. \\
La prima riga contiene tre interi: N, M, S; rispettivamente il numero degli studenti, il numero dei bug e la massima quantità di punti che Tarboto può distribuire alla prossima verifica.\\ 
La seconda riga contiene M interi separati da uno spazio: $C_i$ è la complessità dell'i-esimo bug.\\
La terza riga contiene N interi separati da uno spazio: $A_i$ è l'abilità dell'i-esimo studente. \\
La quarta riga contiene N interi separati da uno spazio: $P_i$ è la quantità di punti che l'i-esimo studente chiede in cambio del proprio lavoro.

\section*{Dati di output}
  
La prima riga dell'output deve contenere la stringa "SI" (senza virgolette e senza accento) oppure "NO" (senza virgolette), a seconda se gli studenti del professor Tarboto riusciranno a risolvere tutti i problemi. \\
Se la risposta è "SI", l'output deve contenere una seconda riga costituita da M interi separati da spazio. L'i-esimo di questi numeri deve essere il numero dello studente che corregge l'i-esimo bug nella soluzione ottimale. \\
Se ci sono più soluzioni ottimali, è sufficiente stamparne una qualsiasi.
  \section*{Assunzioni}
\begin{itemize}
\item 
$1 \leq N, M \leq 10^5$
\item
$0 \leq S \leq 10^9$
\item
$1 \leq C_i, A_i \leq 10^9$
\item
$0 \leq P_i \leq 10^9$
\end{itemize}
 
\noindent Nel 50\% dei casi vale che $N, M, S, A_i, C_i, P_i \leq 100$.

\section*{Nota importante per la valutazione}
Soluzioni che stampano \textbf{solo} la prima riga di output ("SI" o "NO") verranno valutate con il 30\% dei punti per ogni caso di prova.

\section*{Esempi di input/output}

  
    \noindent
    \begin{tabular}{p{8cm}|p{8cm}}
    \toprule
    \textbf{Input \texttt{(stdin)}}
    & \textbf{Output \texttt{(stdout)}}
    \\
    \midrule
    \small
    \begin{verbatim}
3 4 9
1 3 1 2
2 1 3
4 3 6

      \end{verbatim}
    &
    \small
    \begin{verbatim}
SI
2 3 2 3
      \end{verbatim}
    \\
    \bottomrule
    \end{tabular}
\\[12pt]
\textbf{Chiarificazione dell'esempio} \\
Il terzo studente (con abilità 3) risolve il secondo e il quarto bug (con complessità 3 e 2, rispettivamente) e il secondo studente (con abilità 1) risolve il primo e il terzo bug (le loro complessità sono 1). \\ Risolvere ciascun bug richiede un giorno per ciascuno studente, quindi i giorni totali sono 2 (gli studenti possono lavorare in parallelo).\\
Il secondo studente vuole 3 punti, il terzo 6 punti e ciò incontra la disponibilità del professore che è pronto a distribuire al massimo 9 punti.
\end{document}

