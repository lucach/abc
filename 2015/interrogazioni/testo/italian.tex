\usepackage{xcolor}
\usepackage{afterpage}
\usepackage{pifont,mdframed}
\usepackage[bottom]{footmisc}
\usepackage{hyperref}

\makeatletter
\gdef\this@inputfilename{input.txt}
\gdef\this@outputfilename{output.txt}
\makeatother

\newcommand{\inputfile}{\texttt{input.txt}}
\newcommand{\outputfile}{\texttt{output.txt}}

\newenvironment{warning}
  {\par\begin{mdframed}[linewidth=2pt,linecolor=gray]%
    \begin{list}{}{\leftmargin=1cm
                   \labelwidth=\leftmargin}\item[\Large\ding{43}]}
  {\end{list}\end{mdframed}\par}

    La commissione istituita per l'esame di maturità della classe di Luca ha ricevuto dal MIUR l'elenco dei codici SIDI di tutti gli $N$ alunni da esaminare, con il suggerimento che l'ordine delle interrogazioni all'esame orale segua, in ordine crescente, il codice SIDI. Ciò significa che uno studente $i$ dovrebbe essere interrogato prima di uno studente $j$ se il codice SIDI del primo è inferiore a quello del secondo (ovvero $C_i < C_j$).

    Seguire le indicazioni del Ministero comporterebbe però una certa confusione tra gli studenti, abituati a essere interrogati in ordine per numero crescente di registro (numero 1, numero 2, \ldots, numero $N$).

    Assecondando le richieste degli studenti, la commissione stabilisce di adottare la modalità consueta (in ordine per numero crescente di registro), ma vuole prima stabilire qual è l'impatto di questa scelta. In particolare si vuole stabilire per quante coppie di studenti $(i, j)$ cambierebbe l'ordine di interrogazione, ovvero per quante coppie $(i, j)$ $i$ verrebbe interrogato prima di $j$ con uno dei due metodi ma dopo $j$ con l'altro.

\InputFile
Il file \inputfile{} è composto da 2 righe. La prima riga contiene l'intero $N$, il numero di studenti.
La seconda riga contiene $N$ interi positivi separati da spazio: i codici SIDI degli alunni dal numero 1 al numero N.

\OutputFile
Il file \outputfile{} contiene un singolo intero: il \textbf{numero di coppie} di studenti che cambiano l'ordine relativo di interrogazione tra il primo e il secondo metodo.

\begin{warning}
    Attenzione: la risposta al problema supera, in alcuni casi, $2^{32}$. Se si lavora, ad esempio, in C/C++ è quindi richiesto l'uso del tipo \texttt{long long} al posto di \texttt{int}.
\end{warning}


% Assunzioni
\Constraints
\begin{itemize}[nolistsep, itemsep=2mm]
    \item $2 \le N \le 200\,000$.
    \item $1 \le C_i \le 10\,000\,000$ per ogni $i=0\ldots N-1$.
    \item Il codice SIDI è univoco: $C_i \ne C_j$ per ogni $(i, j)$ con $i \ne j$.
\end{itemize}

\pagebreak
\Scoring
Il tuo programma verrà testato su diversi test case raggruppati in subtask.
Per ottenere il punteggio relativo ad un subtask, è necessario risolvere
correttamente tutti i test relativi ad esso.

\begin{itemize}[nolistsep,itemsep=2mm]
  \item \textbf{\makebox[2cm][l]{Subtask 1} [10 punti]}: Casi d'esempio.
  \item \textbf{\makebox[2cm][l]{Subtask 2} [40 punti]}: $N \leq 1\,000$
  \item \textbf{\makebox[2cm][l]{Subtask 3} [25 punti]}: Gli $N$ studenti hanno un codice SIDI compreso tra $1$ e $N$ (ovvero $1 \le C_i \le N$ per ogni $i=0\ldots N-1$).
  \item \textbf{\makebox[2cm][l]{Subtask 4} [25 punti]}: Nessuna limitazione specifica.
\end{itemize}

% Esempi
\Examples
\begin{example}
\exmp{
4
14 63 22 31
}{%
2
}%
\end{example}
\begin{example}
\exmp{
5
84 31 57 25 66
}{%
6
}%
\end{example}

\Explanation
Nel primo esempio considera le seguenti coppie di studenti il cui ordine cambia tra i due metodi:
\begin{itemize}
    \item Codici SIDI 63 e 22 (63 > 22), sono rispettivamente i numeri 2 e 3 nel registro (2 < 3).
    \item Codici SIDI 63 e 31 (63 > 31), sono rispettivamente i numeri 2 e 4 nel registro (2 < 4).
\end{itemize}
Tutte le altre coppie mantengono la stessa posizione relativa: gli studenti della coppia verranno interrogati nello stesso ordine indipendentemente dal metodo utilizzato.
