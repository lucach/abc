\usepackage{xcolor}
\usepackage{afterpage}
\usepackage{pifont,mdframed}
\usepackage[bottom]{footmisc}
\usepackage{hyperref}

\makeatletter
\gdef\this@inputfilename{input.txt}
\gdef\this@outputfilename{output.txt}
\makeatother

\newcommand{\inputfile}{\texttt{input.txt}}
\newcommand{\outputfile}{\texttt{output.txt}}

\newenvironment{warning}
  {\par\begin{mdframed}[linewidth=2pt,linecolor=gray]%
    \begin{list}{}{\leftmargin=1cm
                   \labelwidth=\leftmargin}\item[\Large\ding{43}]}
  {\end{list}\end{mdframed}\par}

    Luca sta ripassando matematica in vista dell'esame di maturità con un suo compagno di classe, Matteo. Dopo un po' di tempo però, presi dalla noia, cominciano a scarabocchiare qualche numero e qualche figura geometrica a caso e si inventano al momento un gioco interessante.

    Il gioco è stato così deciso dai due amici: Matteo scarabocchia $N$ numeri sul foglio e Luca deve cercare di sceglierne tre in modo che \textbf{NON} possano rappresentare le lunghezze dei tre lati di un triangolo valido. Se Luca riesce a trovare anche solo una combinazione ha vinto, altrimenti ha vinto Matteo.

    \begin{warning}
        Tre numeri sono lunghezze valide per i lati di un triangolo se ciascuno è minore o uguale alla somma degli altri due.
    \end{warning}
    \begin{warning}
        In questa situazione viene considerato triangolo anche un cosiddetto ``triangolo degenere'' (ovvero quando un lato misura quanto la somma degli altri due).
    \end{warning}

    Sapresti contare quante possibili combinazioni Luca potrebbe trovare?


\InputFile
Il file \inputfile{} è composto da due righe. La prima riga contiene l'intero $N$. La seconda riga contiene $N$ interi separati da uno spazio, i numeri $V_i$ che Matteo ha scritto sul foglio.

\OutputFile
Il file \outputfile{} è composto da un'unica riga contenente un unico intero, la risposta a questo problema.

% Assunzioni
\Constraints
\begin{itemize}[nolistsep, itemsep=2mm]
	\item $3 \le N \le 15\,000$.
	\item $1 \le V_i \le 10\,000\,000$ per ogni $i=0\ldots N-1$.
\end{itemize}

\pagebreak
\Scoring
Il tuo programma verrà testato su diversi test case raggruppati in subtask.
Per ottenere il punteggio relativo ad un subtask, è necessario risolvere
correttamente tutti i test relativi ad esso.

\begin{itemize}[nolistsep,itemsep=2mm]
  \item \textbf{\makebox[2cm][l]{Subtask 1} [10 punti]}: Casi d'esempio.
  \item \textbf{\makebox[2cm][l]{Subtask 2} [30 punti]}: $N \leq 1\,000$.
  \item \textbf{\makebox[2cm][l]{Subtask 3} [40 punti]}: $N \leq 5\,000$.
  \item \textbf{\makebox[2cm][l]{Subtask 4} [20 punti]}: Nessuna limitazione specifica.
\end{itemize}

% Esempi
\Examples
\begin{example}
\exmp{
3
1 2 3
}{%
0
}%
\end{example}
\begin{example}
\exmp{
4
3 4 10 6
}{%
2
}%
\end{example}

\Explanation
Nel \textbf{secondo caso di esempio}, le combinazioni con cui NON si ottiene un triangolo sono:
\begin{itemize}[nolistsep, itemsep=1mm]
    \item 3 - 4 - 10
    \item 3 - 6 - 10

