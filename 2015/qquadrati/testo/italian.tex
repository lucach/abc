\usepackage{xcolor}
\usepackage{afterpage}
\usepackage{pifont,mdframed}
\usepackage[bottom]{footmisc}
\usepackage{hyperref}

\makeatletter
\gdef\this@inputfilename{input.txt}
\gdef\this@outputfilename{output.txt}
\makeatother

\newcommand{\inputfile}{\texttt{input.txt}}
\newcommand{\outputfile}{\texttt{output.txt}}

\newenvironment{warning}
  {\par\begin{mdframed}[linewidth=2pt,linecolor=gray]%
    \begin{list}{}{\leftmargin=1cm
                   \labelwidth=\leftmargin}\item[\Large\ding{43}]}
  {\end{list}\end{mdframed}\par}

    Luca sta ripassando matematica in vista dell'esame di maturità e in particolare sta studiando il teorema di Lagrange relativo alle funzioni continue e derivabili. Incuriosito dal lavoro del matematico italiano è andato a cercare quali altri teoremi gli vengono attribuiti e ne ha trovato uno interessante.

    Il teorema in questione è detto ``Teorema dei quattro quadrati'' e afferma che ogni intero positivo può essere espresso come somma di (al più) quattro quadrati perfetti. Formalmente:
    \begin{equation*}
        n = a^2 + b^2 + c^2 + d^2 \ \ \ \forall {n} \in \mathbb{N}
    \end{equation*}
    dove $a$, $b$, $c$, $d$ sono interi non negativi.

    Luca però non riesce a trovare un modo per sapere \italic{in quanti} modi diversi $n$ può essere ottenuto seguendo quanto afferma il teorema. Aiuta Luca!


\InputFile
Il file \inputfile{} è composto da due righe. La prima riga contiene l'intero $N$. La seconda riga contiene $N$ interi separati da uno spazio, i numeri $V_i$ su cui applicare il teorema.

\OutputFile
Il file \outputfile{} è composto da un'unica riga contenente $N$ interi: l'$i$-esimo numero indica \italic{in quanti} modi diversi si può ottenere $V_i$ seguendo il teorema.

% Assunzioni
\Constraints
\begin{itemize}[nolistsep, itemsep=2mm]
	\item $1 \le N \le 200$.
	\item $1 \le V_i \le 2^{15}$ per ogni $i=0\ldots N-1$.
\end{itemize}

\pagebreak
\Scoring
Il tuo programma verrà testato su diversi test case raggruppati in subtask.
Per ottenere il punteggio relativo ad un subtask, è necessario risolvere
correttamente tutti i test relativi ad esso.

\begin{itemize}[nolistsep,itemsep=2mm]
  \item \textbf{\makebox[2cm][l]{Subtask 1} [10 punti]}: Casi d'esempio.
  \item \textbf{\makebox[2cm][l]{Subtask 2} [40 punti]}: $N \leq 30$.
  \item \textbf{\makebox[2cm][l]{Subtask 3} [50 punti]}: Nessuna limitazione specifica.
\end{itemize}

% Esempi
\Examples
\begin{example}
\exmp{
1
25
}{%
3
}%
\end{example}
\begin{example}
\exmp{
2
10 31
}{%
2 2
}%
\end{example}

\Explanation
Nel \textbf{primo caso di esempio}, 25 si può ottenere con le tre seguenti combinazioni:
\begin{itemize}[nolistsep, itemsep=1mm]
    \item $25 = 1^2 + 2^2 + 2^2 + 4^2$
    \item $25 = 3^2 + 4^2 ( + 0^2 + 0^2)$
    \item $25 = 5^2 ( + 0^2 + 0^2 + 0^2)$

